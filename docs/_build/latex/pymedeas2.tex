%% Generated by Sphinx.
\def\sphinxdocclass{report}
\documentclass[letterpaper,10pt,english]{sphinxmanual}
\ifdefined\pdfpxdimen
   \let\sphinxpxdimen\pdfpxdimen\else\newdimen\sphinxpxdimen
\fi \sphinxpxdimen=.75bp\relax
\ifdefined\pdfimageresolution
    \pdfimageresolution= \numexpr \dimexpr1in\relax/\sphinxpxdimen\relax
\fi
%% let collapsible pdf bookmarks panel have high depth per default
\PassOptionsToPackage{bookmarksdepth=5}{hyperref}

\PassOptionsToPackage{warn}{textcomp}
\usepackage[utf8]{inputenc}
\ifdefined\DeclareUnicodeCharacter
% support both utf8 and utf8x syntaxes
  \ifdefined\DeclareUnicodeCharacterAsOptional
    \def\sphinxDUC#1{\DeclareUnicodeCharacter{"#1}}
  \else
    \let\sphinxDUC\DeclareUnicodeCharacter
  \fi
  \sphinxDUC{00A0}{\nobreakspace}
  \sphinxDUC{2500}{\sphinxunichar{2500}}
  \sphinxDUC{2502}{\sphinxunichar{2502}}
  \sphinxDUC{2514}{\sphinxunichar{2514}}
  \sphinxDUC{251C}{\sphinxunichar{251C}}
  \sphinxDUC{2572}{\textbackslash}
\fi
\usepackage{cmap}
\usepackage[T1]{fontenc}
\usepackage{amsmath,amssymb,amstext}
\usepackage{babel}



\usepackage{tgtermes}
\usepackage{tgheros}
\renewcommand{\ttdefault}{txtt}



\usepackage[Bjarne]{fncychap}
\usepackage{sphinx}

\fvset{fontsize=auto}
\usepackage{geometry}


% Include hyperref last.
\usepackage{hyperref}
% Fix anchor placement for figures with captions.
\usepackage{hypcap}% it must be loaded after hyperref.
% Set up styles of URL: it should be placed after hyperref.
\urlstyle{same}


\usepackage{sphinxmessages}
\setcounter{tocdepth}{1}



\title{pymedeas2}
\date{Dec 21, 2021}
\release{2.0.0\sphinxhyphen{}dev}
\author{Jordi Solé, Roger Samsó, Eneko Martin, Enric Alcover}
\newcommand{\sphinxlogo}{\vbox{}}
\renewcommand{\releasename}{Release}
\makeindex
\begin{document}

\pagestyle{empty}
\sphinxmaketitle
\pagestyle{plain}
\sphinxtableofcontents
\pagestyle{normal}
\phantomsection\label{\detokenize{index::doc}}

\chapter{Contents}
\label{\detokenize{index:contents}}

\section{Installation}
\label{\detokenize{installation:installation}}\label{\detokenize{installation::doc}}
\sphinxAtStartPar
For installing \sphinxstyleemphasis{pymedeas2} models you need to first clone or download and unzip the \sphinxhref{https://gitlab.com/gencat\_creaf/pymedeas2}{pymedeas2 repository}.


\subsection{Installing using conda}
\label{\detokenize{installation:installing-using-conda}}\begin{enumerate}
\sphinxsetlistlabels{\arabic}{enumi}{enumii}{}{.}%
\item {} 
\sphinxAtStartPar
If not installed yet, \sphinxhref{https://conda.io/projects/conda/en/latest/user-guide/install/index.html}{download and install Miniconda or Anaconda} on your computer.

\item {} 
\sphinxAtStartPar
Open a terminal or Anaconda Prompt Powershell and navigate to the project folder (see {\hyperref[\detokenize{navigating::doc}]{\sphinxcrossref{\DUrole{doc}{using the terminal or Anaconda Prompt Powershell}}}} for more information).

\item {} 
\sphinxAtStartPar
\sphinxhref{https://docs.conda.io/projects/conda/en/latest/user-guide/tasks/manage-environments.html\#creating-an-environment-from-an-environment-yml-file}{Create a conda environement using the provided environment.yml file}, e.g.:

\end{enumerate}

\begin{sphinxVerbatim}[commandchars=\\\{\}]
(base) user@host:\PYGZti{}/pymedeas2\PYGZdl{} conda env create \PYGZhy{}f environment.yml
\end{sphinxVerbatim}

\begin{sphinxadmonition}{note}{Note:}
\sphinxAtStartPar
If you already had Anaconda installed on your Mac and then upgraded the OS to Catalina, If you are running MacOS Catalina, make sure your read \sphinxhref{https://www.anaconda.com/how-to-restore-anaconda-after-macos-catalina-update/}{this}
\end{sphinxadmonition}


\subsection{Alternative installation}
\label{\detokenize{installation:alternative-installation}}
\sphinxAtStartPar
The models can be also run if the needed dependencies (found in the \sphinxstyleemphasis{environment.yml} file) are required. Alternative package manages such as \sphinxstyleemphasis{pip} can be used to install these dependencies.


\subsection{Required Dependencies}
\label{\detokenize{installation:required-dependencies}}
\sphinxAtStartPar
\sphinxstyleemphasis{pymedeas2} models need \sphinxhref{https://pysd.readthedocs.io}{PySD} library for running. It requires at least \sphinxstylestrong{Python 3.7} and \sphinxstylestrong{PySD 2.2.0}. Moreover \sphinxstylestrong{matplotlib} and \sphinxstylestrong{dacite} are required.


\section{Usage}
\label{\detokenize{usage:usage}}\label{\detokenize{usage::doc}}

\subsection{Running a simulation}
\label{\detokenize{usage:running-a-simulation}}\label{\detokenize{usage:id1}}
\sphinxAtStartPar
The models can be run from command line.
\begin{enumerate}
\sphinxsetlistlabels{\arabic}{enumi}{enumii}{}{.}%
\item {} 
\sphinxAtStartPar
Open a terminal or Anaconda Prompt Powershell, navigate to the project folder, and activate the conda environment (see {\hyperref[\detokenize{navigating::doc}]{\sphinxcrossref{\DUrole{doc}{using the terminal or Anaconda Prompt Powershell}}}} for more information).

\item {} 
\sphinxAtStartPar
At this point, you should be able to run a default simulation calling the \sphinxtitleref{run.py} file with python:

\end{enumerate}

\begin{sphinxVerbatim}[commandchars=\\\{\}]
\PYG{g+gp+gpVirtualEnv}{(pymedeas)} \PYG{g+gp}{user@host:\PYGZti{}/pymedeas2\PYGZdl{} }python run.py
\end{sphinxVerbatim}
\begin{enumerate}
\sphinxsetlistlabels{\arabic}{enumi}{enumii}{}{.}%
\setcounter{enumi}{2}
\item {} 
\sphinxAtStartPar
By default the World model will run, but you can use the \sphinxstyleemphasis{\sphinxhyphen{}m} option to select a nested model, e.g.:

\end{enumerate}

\begin{sphinxVerbatim}[commandchars=\\\{\}]
\PYG{g+gp+gpVirtualEnv}{(pymedeas)} \PYG{g+gp}{user@host:\PYGZti{}/pymedeas2\PYGZdl{} }python run.py \PYGZhy{}m pymedeas\PYGZus{}eu
\end{sphinxVerbatim}

\begin{sphinxadmonition}{note}{Note:}
\sphinxAtStartPar
To see all user options and default parameter values, run the help flag, e.g.:

\begin{sphinxVerbatim}[commandchars=\\\{\}]
\PYG{g+gp+gpVirtualEnv}{(pymedeas)} \PYG{g+gp}{user@host:\PYGZti{}/pymedeas2\PYGZdl{} }python run.py \PYGZhy{}\PYGZhy{}help
\end{sphinxVerbatim}

\sphinxAtStartPar
You can add for example modify the final time, the time step, variable values or the output file name, between others.
\end{sphinxadmonition}
\begin{enumerate}
\sphinxsetlistlabels{\arabic}{enumi}{enumii}{}{.}%
\setcounter{enumi}{3}
\item {} 
\sphinxAtStartPar
After finishing you can continue launching more simulations, {\hyperref[\detokenize{usage:plotting-simulation-results}]{\sphinxcrossref{plotting simulation results}}}, or deactivate your environment or directly close the terminal or Anaconda Powershell Prompt:

\end{enumerate}


\subsection{Model outputs}
\label{\detokenize{usage:model-outputs}}
\sphinxAtStartPar
Simulation results (csv file) can be found either in the respective folder inside the \sphinxstyleemphasis{outputs} folder.

\sphinxAtStartPar
Unless the user provides the desired output file name with the \sphinxhyphen{}n option when launching the simulation (e.g. python run.py \sphinxhyphen{}n results\_my\_scenario), the default results naming convention is the following:

\sphinxAtStartPar
\sphinxstyleemphasis{results\_SCENARIO\sphinxhyphen{}NAME\_INITIAL\sphinxhyphen{}DATE\_FINAL\sphinxhyphen{}DATE\_TIME\sphinxhyphen{}STEP.csv}

\sphinxAtStartPar
If a results file with the same name already exists, the suffix “\_old” will be added at the end of the file name. This can happen up to two times. NOTE that if a fourth simulation with the same name is run, the file of the first simulation result will be automatically deleted.


\subsection{Plotting simulation results}
\label{\detokenize{usage:plotting-simulation-results}}\label{\detokenize{usage:id2}}\begin{enumerate}
\sphinxsetlistlabels{\arabic}{enumi}{enumii}{}{.}%
\item {} 
\sphinxAtStartPar
Open a terminal or Anaconda Prompt Powershell, navigate to the project folder, and activate the conda environment (see {\hyperref[\detokenize{navigating::doc}]{\sphinxcrossref{\DUrole{doc}{using the terminal or Anaconda Prompt Powershell}}}} for more information).

\item {} 
\sphinxAtStartPar
Call the \sphinxtitleref{plot\_tool.py} file with python:

\end{enumerate}

\begin{sphinxVerbatim}[commandchars=\\\{\}]
\PYG{g+gp+gpVirtualEnv}{(pymedeas)} \PYG{g+gp}{user@host:\PYGZti{}/pymedeas2\PYGZdl{} }python plot\PYGZus{}tool.py
\end{sphinxVerbatim}
\begin{enumerate}
\sphinxsetlistlabels{\arabic}{enumi}{enumii}{}{.}%
\setcounter{enumi}{2}
\item {} 
\sphinxAtStartPar
Simulation results can be found either in the respective folder inside the \sphinxstyleemphasis{outputs} folder. You can load an unlimited number of results files, to compare several simulation results.

\end{enumerate}


\section{Developer Documentation}
\label{\detokenize{development/development_index:developer-documentation}}\label{\detokenize{development/development_index::doc}}
\sphinxAtStartPar
Development documentation for improving the \sphinxstyleemphasis{pymedeas2} workflow and adding new models.


\subsection{Adding models}
\label{\detokenize{development/adding_models:adding-models}}\label{\detokenize{development/adding_models::doc}}

\section{Using the terminal or Anaconda Prompt Powershell}
\label{\detokenize{navigating:using-the-terminal-or-anaconda-prompt-powershell}}\label{\detokenize{navigating::doc}}
\sphinxAtStartPar
In order to install, run the models, and plot the results a command\sphinxhyphen{}line interface (CLI) is used. For Linux and MacOS user the Terminal is recommended, while for Windows used the Anaconda Prompt Powershell is recommended (installed together with miniconda or Anaconda).


\subsection{Navigate to the project folder}
\label{\detokenize{navigating:navigate-to-the-project-folder}}
\sphinxAtStartPar
You can navigate in the terminal (Linux/MacOS) or Anaconda Powershell Prompt (Windows) using the \sphinxstyleemphasis{cd} command, example in Linux terminal:

\begin{sphinxVerbatim}[commandchars=\\\{\}]
(base) user@host:\PYGZti{}\PYGZdl{} cd pymedeas2
(base) user@host:\PYGZti{}/pymedeas2\PYGZdl{}
\end{sphinxVerbatim}

\sphinxAtStartPar
or in Anaconda Powershell:

\begin{sphinxVerbatim}[commandchars=\\\{\}]
\PYG{p}{(}\PYG{n}{base}\PYG{p}{)} \PYG{n}{PS} \PYG{n}{C}\PYG{p}{:}\PYGZbs{}\PYG{n}{Users}\PYGZbs{}\PYG{n}{user}\PYG{o}{\PYGZgt{}} \PYG{n}{cd} \PYG{n}{pymedeas2}
\PYG{p}{(}\PYG{n}{base}\PYG{p}{)} \PYG{n}{PS} \PYG{n}{C}\PYG{p}{:}\PYGZbs{}\PYG{n}{Users}\PYGZbs{}\PYG{n}{user}\PYGZbs{}\PYG{n}{pymedeas2}\PYG{o}{\PYGZgt{}}
\end{sphinxVerbatim}

\begin{sphinxadmonition}{note}{Note:}
\sphinxAtStartPar
The current path position is shown in the left side of the command line. If you need to to show the files and folder inside the current position you can use \sphinxcode{\sphinxupquote{ls}} command:

\begin{sphinxVerbatim}[commandchars=\\\{\}]
\PYG{g+gp+gpVirtualEnv}{(base)} \PYG{g+gp}{user@host:\PYGZti{}/pymedeas2\PYGZdl{} }ls
\end{sphinxVerbatim}
\end{sphinxadmonition}

\sphinxAtStartPar
You need to know where exactly you have downloaded the models directory, and navigate until there. You can use \sphinxcode{\sphinxupquote{cd ..}} to go one folder back.


\subsection{Activating and deactivating conda environment}
\label{\detokenize{navigating:activating-and-deactivating-conda-environment}}
\sphinxAtStartPar
Once the conda environment has been installed, you can \sphinxhref{https://docs.conda.io/projects/conda/en/latest/user-guide/tasks/manage-environments.html\#activating-an-environment}{activate the conda environment} from any position in the terminal or Anaconda Powershell Prompt, e.g.:

\begin{sphinxVerbatim}[commandchars=\\\{\}]
\PYG{g+gp+gpVirtualEnv}{(base)} \PYG{g+gp}{user@host:\PYGZti{}/pymedeas2\PYGZdl{} }conda activate pymedeas
\PYG{g+gp+gpVirtualEnv}{(pymedeas)} \PYG{g+gp}{user@host:\PYGZti{}/pymedeas2\PYGZdl{}}
\end{sphinxVerbatim}

\begin{sphinxadmonition}{note}{Note:}
\sphinxAtStartPar
When activated you will see that the \sphinxstyleemphasis{(base)} on the left\sphinxhyphen{}hand side has become a \sphinxstyleemphasis{(pymedeas)}.
\end{sphinxadmonition}

\sphinxAtStartPar
Once finish working, if you want to continue using the CLI you can \sphinxhref{https://docs.conda.io/projects/conda/en/latest/user-guide/tasks/manage-environments.html\#deactivating-an-environment}{deactivate the conda environment}, e.g.:

\begin{sphinxVerbatim}[commandchars=\\\{\}]
\PYG{g+gp+gpVirtualEnv}{(pymedeas)} \PYG{g+gp}{user@host:\PYGZti{}/pymedeas2\PYGZdl{} }conda deactivate
\PYG{g+gp+gpVirtualEnv}{(base)} \PYG{g+gp}{user@host:\PYGZti{}/pymedeas2\PYGZdl{}}
\end{sphinxVerbatim}

\sphinxAtStartPar
Otherwise you can directly close the terminal or Anaconda Powershell Prompt.



\renewcommand{\indexname}{Index}
\printindex
\end{document}
