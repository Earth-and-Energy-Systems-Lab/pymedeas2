The MEDEAS models dynamically operate as follows: for each period, a sectoral economic demand is estimated from exogenous pathways of expected Gross Domestic Product per capita  (GDPpc) and population evolution. The final energy demand required to fulfil production is obtained using energy-economy hybrid input-output analysis, and energy intensities by type of final energy. The energy sub-module computes the available final energy supply, which may or may not satisfy demand, adapting the economic production to the available energy. The materials required by the economy, with emphasis on those required by alternative green technologies, are estimated; this allows to assess eventual future mineral bottlenecks. The new energy infrastructure requires energy investments, whose computation allows to estimate the variation of the EROI (Energy Return over Energy Invested) of the system, which in turn affects the final energy demand. The climate submodule computes the greenhouse gas (GHG) emissions associated to the resulting energy mix (complemented by exogenous pathways for non-energy emissions), which feeds back to the economy, affecting final demand. Additional land requirements are accounted for. Finally, the social and environmental impacts are computed. For more detail the reader is referred to Refs. --> **all this is taken from [@SAMSO2020100582] and needs to be either cited or modified**


\urgent{Update the general logic of pymedeas}

