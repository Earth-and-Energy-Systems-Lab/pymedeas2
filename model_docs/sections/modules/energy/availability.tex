\note{Oil extraction, reformulate in only one way}
\note{Gas extraction, reformulate in only one way}
\note{Explain year scarcity}
\note{Explain nuclear}


The availability of non-renewable energy sources is calculated by each of the sources included in the model: oil, natural gas, coal and uranium.
The model considers the following non-renewable energy resources:

\begin{enumerate}
    \item Conventional oil: refers to crude oil and NGLs.
    \item Unconventional oil: includes heavy and extra-heavy oil, natural bitumen (oil sand and tar
    sands) and oil shales. Biofuels, CTL, GTL and refinery gains are modeled separately.
    \item Conventional gas.
    \item Unconventional gas: includes shale gas, tight gas, coal-bed methane (CBM) and hydrates.
    \item Coal: includes anthracite, bituminous, sub-bituminous, black, brown and lignite coal.
    \item Uranium
\end{enumerate}

Nuclear fusion is not considered since the ITER and DEMO projects
estimate that the first commercial fusion power would not be available before 2040, which would prevent this technology to substantially contribute to the mix
in the timeline of MEDEAS.

The availability of non-renewable energy resources in MEDEAS depends upon two constraints:

\begin{enumerate}
    \item Stock: available resource in the ground
    \item Flow: extraction rate of this resource
\end{enumerate}

The available stock of a resource is usually measured in terms of ultimately recoverable resources
($URR$), or remaining RURR ($RURR$) if referenced to a given year. The RURR in a given time t is defined
as the difference between the URR and cumulative extraction ($CExt$) in time t:

In order to estimate the future availability of fossil fuels, we have reviewed the studies providing
depletion curves for non-renewable energy resources taking into account both stocks and flow
limits. 

The depletion curves of non-renewable energies reviewed in the literature represent extraction
levels compatible with geological constraints as a function of time. As the model is demand based, we assume that, while the maximum extraction rate is not reached, the extraxction of each resource matches the demand. Actual
extraction will therefore be the minimum between the demand and the maximum extraction rate. To do this, the depletion curves have been converted into maximum production
curves as a function of remaining resources. In these curves, as long as the remaining resources are
large, extraction is only constrained by the maximum extraction level. However, with cumulated
extraction, there is a level of remaining resources when physical limits start to appear and maximum
extraction rates are gradually reduced. In this way, the model uses a stock of resources (the RURR)
and it studies how this stock is exhausted depending on production, which is in turn determined by
demand and maximum extraction.

\begin{equation}
    RURR_t=URR - CExt_t
    \label{eq:available_stock}
\end{equation}
    
\enric{Add  Literature review of depletion curves by fuel}

\paragraph{Oil Extraction}
The limitation of the oil resources, and availability of oil depends on two constraints: the stock (EJ) and the flows (Watts). The model priorizes the others NRE liquids fuels, then the oil demand is obtained as follows:

\begin{equation}
Oil_{demand}=PED_{NRE\_Liquids}-FES_{CTL}- FES_{GTL}- ORF
\label{eq:oil-demand}
\end{equation}

Where $PED\_NRE\_Liquids$ is the Primary energy demand of non-renewable eneregy liquids (EJ), $FES\_GTL$ is the final energy demand of GTL (Gas-to-liquids) (EJ),  FES\_CTL is the final energy demand of GTL (Coal-to-liquids) (EJ) and ORF is the oil refinery gains.

\paragraph{Natural gas extraction}

\paragraph{Coal extraction}

Coal extraction submodule is in charge of obtaining the coal extraction taking into account the limitations of the coal resources. The model priorizes other solid NRE sources for satisfying the primary energy demand.

\begin{equation}
Coal_{demand}=PED_{solids}-PE_{trad\_bio}-PES_{peat}-PES_{waste}-LCP
\label{eq:coal-demand}
\end{equation}

Where, $PED\_solids$ is the primary energy demand of the solids (EJ), $PE\_trad\_bio$ is the primary energy of the traditional biomass (EJ), $PES\_peat$ is the Primary energy suplly of peat (EJ), PES\_waste is the primary energy obtained from waste (EJ) and LCP are the losses in charcoal plants (EJ).

The amount of coal that can be extracted is limited by the Hubbert curves if the parameters *unlimited coal?* and *unlimited NRE?* are desactivated. Then, the extraction of coal is limited by the maximum extraction limit as:

\begin{equation}
Coal_{extraction}=min(PED_{coal}, max\_extract\_coal))
\label{eq:coal-extraction}
\end{equation}

\paragraph{Uranium extraction}

Uranium extraction depends on demand and resource availability, taking into account the Hubbert curve of maxim extraction of the uranium. The demand of uranium is obtained from the potential generation of nuclear electricity divided by the efficiency of uranium for electricity. If the parameters *unlimited uranium?* and *unlimited NRE?* are desactivated, the extraction of uranium is limited by the maximum extraction limit as:
\eneko{Avoid talking about parameters like unlimited uranium or unlimited NRE, they are used for research but their non-default value has no physical sense!}

\begin{equation}
Uranium_{extraction}=min(PED_{uranium}, max\_extract\_uranium))
\label{eq:uranium-extraction}
\end{equation}

In the nested models, for calculating the abundance of uranium, the imports from the rest of the world are included. In such way, when there is local scarcity, the imports can compensate it.

\warning{There can be a double counting of imports of the RoW of the nested model of Europe and the nested model of Catalonia!}

\paragraph{Final energy abundance}

The abundance of final fuels is obtained from the primary energy demand (PED) and the primary energy supply (PES) as follows:

\begin{equation}
A= \frac{PED_i-PES_i}{PED_i}
\label{eq:abundance-final-fuels}
\end{equation}

When the $PED<PES$ the abundance is always 1. The index i is the energy carrier: liquids, gases, solids, electricity and heat.
Then, depending on the \emph{sensitivity to scarcity option} defined in the scenarios files, the perception of final energy scarcity of each fuel by economic sectors is obtained. This perception drives the fuel replacement and efficiency improvement. This perception of scarcity decreases on time depending on the \emph{energy scarcity forgeting time} defined also at the scenario files as the time in years that society takes to forget the percepticon of scarcity for economic sectors.

\paragraph{Renewable energy sources (RES) availability}

Although renewable energy is usually considered a huge abundant source of energy, there are technological and ecological limits to its development. . However, the large scale deployment of renewable alternatives faces serious challenges in relation to their integration in the electricity mix due to their intermittency, seasonality and uneven spatial distribution requiring storage, their lower energy density, their dependence on minerals and materials for the construction of power plants and related infrastructures and their associated environmental impacts, that pose problems on its fast and global growth. All this factors significantly reduce their sustainable potential.

\subparagraph{Bioenergy}
Bioenergy provides approximately 10\% of global primary energy supply and is produced from a set of sources (dedicated crops, residues and Municipal Solid Waste (MSW), etc.) that can serve different uses (biofuels, heat, electricity, etc.), although traditional biomass use dominates. 

We model bioenergy in 4 main categories: traditional biomass, conventional solid biomass, dedicated crops and residues. Peatlands are the most efficient terrestrial ecosystems in storing carbon. Degradation of peatlands is a major and growing source of anthropogenic greenhouse gas emissions. Peatlands are important natural ecosystems with high value for biodiversity conservation, climate regulation and human welfare \cite{Parish2008}. For these reasons, this energy source is not suposed to contribute to a sustainable energy mix in the future.
 
Since bioenergy can be used for different final uses (heat, electricity, solids, biofuels), a number of assumptions in relation to the use of the potential need to be made to run the model.

\textbf{Uses of bioenergy}

\enric{Revise}

\begin{enumerate}
    \item \underline{Traditional biomass:} It is the biomass used by large populations in poor-countries. There is much uncertainty around the amount of traditional biomass currently used. We asume the consumption ratio constant over time (0.29 toe per capita). The demand of traditional biomass in pymedeas is driven by the demand of solids by the households (IOTs).
    \item \underline{Conventional solid biomass:} refers to modern uses of solid biomass for heat and electricity, excluding plantations in marginal lands and residues, i.e. mainly from tree plantations. Since current conventional modern bioenergy use for heat and electricity (18+4 EJ/yr harvestable NPP respectively (IEA, 2016a; REN21, 2016)) already surpasses sustainable levels we (optimistically) assume that in the future better practices could be adopted allowing to increase the sustainable potential to 25-30 EJ/yr (NPP harvestable). An eventual reduced dependence on traditional biomass in the next decades might also allow to use bioenergy resources in a more sustainable way, although this would be limited by the fact that most of the traditional biomass is infact extracted in an unsustainable way.
    \item \underline{Dedicated corps:} in marginal lands and land subject to competition with other uses. Marginal land use refers to lands whose use does not reduce food security, remove forests or endanger conservation lands (Field et al., 2008). We assume that these dedicated crops for bioenergy will be mainly used for biofuel production as it currently the case (2nd -current bioethanol and biodiesel) and given that previous work found that liquids would likely be the first final energy source to face scarcity. It is assumed that the 3rd generation biofuels (cellulosic) do not require additional land, but instead substitute the 2nd generation when the technology is available at a rate depending on the scenario. We assume an improvement of +15\% in the power density in relation to the 2nd generation \cite{WBGU2009}.
    \item \underline{Residues} (agricultural, forestry, municipal, industry, etc.): Currently, only biogas and MSW exist at commercial level. Biogas potential is assumed to focus on the promotion of small plants for agricultural and industrial residues, as well as animal dung which provide major ecological cobenefits \cite{WBGU2009}. Current final use share and efficiencies are assumed constant given its past evolution. The 3rd generation biofuels (cellulosic) are still in R\&D and doesn’t appear in the standard version of the model before 2025 as suggested by the literature (Janda et al., 2012). By-default, residues potential are assigned mostly (75\%) for generating heat and electricity, as it currently happens (IPCC,2007a, 2007b), the rest being used for biofuels production (although this parameter can be modified by the user). There is currently a controversial debate about the potential of the valuation of agricultural and forestry residues, because of its threat to soil fertility preservation in the long run, biodiversity conservation and ecosystem services (Gomiero et al., 2010; Wilhelm et al.,2007). We take the estimation of \cite{WBGU2009} of 25 EJ NPP taking into account economic restrictions. However, it should be kept in mind that that this potential will tend to be progressively degraded by time.
\end{enumerate}


\enric{Include bibliography (A lot of information at the deliverable 4.1)}

\enric{Detail all RES availability}
