\note{Oil extraction, reformulate in only one way}
\note{Gas extraction, reformulate in only one way}
\note{Explain year scarcity}
\note{Explain nuclear}


This submodule calculates the availability of non-renewable energy sources: oil, natural gas, coal and uranium.
The model considers the following non-renewable energy resources:

\begin{enumerate}
    \item Conventional oil: refers to crude oil and NGLs.
    \item Unconventional oil: includes heavy and extra-heavy oil, natural bitumen (oil sand and tar
    sands) and oil shales. Biofuels, CTL, GTL and refinery gains are modeled separately.
    \item Conventional gas.
    \item Unconventional gas: includes shale gas, tight gas, coal-bed methane (CBM) and hydrates.
    \item Coal: includes anthracite, bituminous, sub-bituminous, black, brown and lignite coal.
    \item Uranium
\end{enumerate}

Nuclear fusion is not considered since the ITER and DEMO projects
estimate that the first commercial fusion power would not be available before 2040, which would prevent this technology to substantially contribute to the mix
in the timeline of MEDEAS.

The availability of non-renewable energy resources in MEDEAS depends upon two constraints:

\begin{enumerate}
    \item Stock: available resource in the ground
    \item Flow: extraction rate of this resource
\end{enumerate}

The available stock of a resource is usually measured in terms of ultimately recoverable resources
($URR$), or remaining RURR ($RURR$) if referenced to a given year. The RURR in a given time t is defined
as the difference between the URR and cumulative extraction ($CExt$) in time t:

In order to estimate the future availability of fossil fuels, we have reviewed the studies providing
depletion curves for non-renewable energy resources taking into account both stocks and flow
limits. 

The depletion curves of non-renewable energies reviewed in the literature represent extraction
levels compatible with geological constraints as a function of time. As the model is demand based, we assume that, while the maximum extraction rate is not reached, the extraxction of each resource matches the demand. Actual
extraction will therefore be the minimum between the demand and the maximum extraction rate. To do this, the depletion curves have been converted into maximum production
curves as a function of remaining resources. In these curves, as long as the remaining resources are
large, extraction is only constrained by the maximum extraction level. However, with cumulated
extraction, there is a level of remaining resources when physical limits start to appear and maximum
extraction rates are gradually reduced. In this way, the model uses a stock of resources (the RURR)
and it studies how this stock is exhausted depending on production, which is in turn determined by
demand and maximum extraction.

\begin{equation}
    RURR_t=URR - CExt_t
    \label{eq:available_stock}
\end{equation}
    

\paragraph{Oil Extraction}
The limitation of the oil resources, and availability of oil depends on two constraints: the stock (EJ) and the flows (Watts). The model priorizes the others NRE liquids fuels, then the oil demand is obtained as follows:

\begin{equation}
Oil_{demand}=PED_{NRE\_Liquids}-FES_{CTL}- FES_{GTL}- ORF
\label{eq:oil-demand}
\end{equation}

Where $PED\_NRE\_Liquids$ is the Primary energy demand of non-renewable eneregy liquids (EJ), $FES\_GTL$ is the final energy demand of GTL (Gas-to-liquids) (EJ),  FES\_CTL is the final energy demand of GTL (Coal-to-liquids) (EJ) and ORF is the oil refinery gains.

\paragraph{Natural gas extraction}

\paragraph{Coal extraction}

Coal extraction submodule is in charge of obtaining the coal extraction taking into account the limitations of the coal resources. The model priorizes other solid NRE sources for satisfying the primary energy demand.

\begin{equation}
Coal_{demand}=PED_{solids}-PE_{trad\_bio}-PES_{peat}-PES_{waste}-LCP
\label{eq:coal-demand}
\end{equation}

Where, $PED\_solids$ is the primary energy demand of the solids (EJ), $PE\_trad\_bio$ is the primary energy of the traditional biomass (EJ), $PES\_peat$ is the Primary energy suplly of peat (EJ), PES\_waste is the primary energy obtained from waste (EJ) and LCP are the losses in charcoal plants (EJ).

The amount of coal that can be extracted is limited by the Hubbert curves if the parameters *unlimited coal?* and *unlimited NRE?* are desactivated. Then, the extraction of coal is limited by the maximum extraction limit as:

\begin{equation}
Coal_{extraction}=min(PED_{coal}, max\_extract\_coal))
\label{eq:coal-extraction}
\end{equation}

\paragraph{Uranium extraction}

Uranium extraction depends on demand and resource availability, taking into account the Hubbert curve of maxim extraction of the uranium. The demand of uranium is obtained from the potential generation of nuclear electricity divided by the efficiency of uranium for electricity. If the parameters *unlimited uranium?* and *unlimited NRE?* are desactivated, the extraction of uranium is limited by the maximum extraction limit as:
\eneko{Avoid talking about parameters like unlimited uranium or unlimited NRE, they are used for research but their non-default value has no physical sense!}

\begin{equation}
Uranium_{extraction}=min(PED_{uranium}, max\_extract\_uranium))
\label{eq:uranium-extraction}
\end{equation}

In the nested models, for calculating the abundance of uranium, the imports from the rest of the world are included. In such way, when there is local scarcity, the imports can compensate it.

\warning{There can be a double counting of imports of the RoW of the nested model of Europe and the nested model of Catalonia!}

\paragraph{Final energy abundance}

The abundance of final fuels is obtained from the primary energy demand (PED) and the primary energy supply (PES) as follows:

\begin{equation}
A= \frac{PED_i-PES_i}{PED_i}
\label{eq:abundance-final-fuels}
\end{equation}

When the $PED<PES$ the abundance is always 1. The index i is the energy carrier: liquids, gases, solids, electricity and heat.
Then, depending on the \emph{sensitivity to scarcity option} defined in the scenarios files, the perception of final energy scarcity of each fuel by economic sectors is obtained. This perception drives the fuel replacement and efficiency improvement. This perception of scarcity decreases on time depending on the \emph{energy scarcity forgeting time} defined also at the scenario files as the time in years that society takes to forget the percepticon of scarcity for economic sectors.