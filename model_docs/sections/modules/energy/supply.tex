Primary total energy demand is covered with different Primary Energy Sources (PES) gruped in five categories: solids, liquids, gases, electricity and heat



% \begin{table}
%     \textbf{MEDEAS final energy category} ; \textbf{NRE / RES} ; \textbf{Energy source modelled in MEDEAS}
% \end{table}
    

\paragraph{Nuclear}

Installation of nuclear plants is limited by several factors: uranium availability, RES supply, nuclear Cp.

There are four different scenarios
\begin{enumerate}
    \item Constant power at current levels
    \item No more nuclear power installed capacity
    \item Growth of nuclear power installed capacity
    \item Phase-out nuclear power
\end{enumerate}

In the firts one, the nuclear capacity is being substitute when the lifetime of the installations is reached; in the second scenario, there is not more nuclear power installed when the facilities are depreciated; in the third scenario there is a yearly increase of the power capacity; and in the phase-out scenario there is a decrease of nuclear facilities before its lifetime is reached.

The new annual increase of new planned nuclear capacity is zero except for the scenario 3, where this increase of capacity is defined in the scenario's excel. Then, the new requiered capacity is obtained by:

\begin{equation}
C_{new \_nuc}=   C_{nuc} \cdot G_{nuc\_elect} \cdot Ef_{scarcity\_uranium} \cdot Cp_{limit}
\label{eq:new-nuc-capacity}
\end{equation}

Where $C\_num$ is the actual installed capacity, $G\_nuc\_elect$ is the annual growth of new planned nuclear capacity, $Ef_{scarcity\_uranium}$ is the efect of uranium scarcity and Cp limit is a factor that limitates the new capacity installation when the capacity factor of nuclear due to the RES penetration falls below 60\%.

The efects of the uranium scarcity, that is calculated in the availabilty submodule, con be modeled by a relationship that avoids an abrupt limitation by introducing a range (1;0.8) in the uranium abundance that constrains the development of new nuclear facilities. This relation models the behaviour of new required nuclear capacity when the abundance is in the range (1;0.8):

\begin{equation}
Ef_{scarcity\_uranium}=  ((A_{Uranium}-0.8) \cdot 5)^2
\label{eq:effects-uranium-scarcity}
\end{equation}

Then, the new planed nuclear capacity is calculated from the new nuclear capacity under planning plus replacement nuclear capacity less nuclear capacity under construction.

\paragraph{Solar potentials in urban areas}


\paragraph{RES electricity potentials}

Nine types of RES for electricity generation are modeled: hydro, solar PV, solar CSP, onshore wind, offshore wind, geothermal, biomass, oceanic and biogas.

Renewables potentials are limited by biophysical sustainable constraints, limiting the potential installed by a techno-sustainable limits that are taking into account the ecological, and the technical limitation of renewable energy sources. This potential limits are read from the "energy.xslx" excel file.

\paragraph{RES electric capacities and generation}

Renewable Energy Source (RES) capacity from different sources (hydro, geothermal, solid biomass, oceanic, wind onshore, wind offshore and solar) is limited by the techno-ecological potentials and its growth is defined in the "scenario.xslx" file, with objectives policies of total capacity installed by each RES every 5 years.  
Then, this desired capacity is limited by the techno-ecological potential, that is defined at "energy.xlsx" file and with values obtained from literature. 

Energy output of the installed capacity is afected by two main sources: the capacity factor and curtailment. Capacity factor (Cp) variation has been discussed in literature, some sources, as \cite{IEA2022}, proposed increasing values of Cp as RES penetration level increases. Others, as \cite{NREL2012}, proposed a decreasing Cp proportional to the renewable share in the electricity generation mix. In our model, a constant Cp obtained from historical data is used. 

The other limiting factor is curtailment. Increased variable RES penetration levels may drive a system to encounter transmission or operational constraints, forcing the system operator to accept less wind or solar than is available, in this cases, curtailment takes part of the electricity generated. As this factor depends on several parameters \cite{Bird2016} as  infrastructural, operational, or institutional development of the grid, as well as the development of an storage technology that can manage the peak generation electricity, it depends on several policies and strategies. This parameter is defined at the scerario's file as a share of total potential renewable electricity generated. 
 
The module prioritize the use of biogas and waste as energy sources, and only after consuming this energy, other RES sources are taked into account.

\enric{Revisar}


\paragraph{RES electric supply by technology}

There is a priority in RES, first of all, electricity generation from waste and bioenergy is calculated and the final electricity demand after priorities obtained as:

\begin{equation}
    FED_{elect\_after\_prior}= FED_{elect\_tot}-FES_{elect\_bio}-FES_{elect\_waste}
    \label{eq:FED-after-priorities}
\end{equation}

Then, $FED_{elect\_after\_prior}$ is used by the RES electricity potential submodule for determining the amount of RES potential to be installed.
Once the model has calculated the total amount of electricity generated from RES, it obtains the electricity demand from NRE that is needed to satisfy the total electricity demand as:

\begin{equation}
    FED_{elect\_NRE}= FED_{elect\_tot}-FES_{RES}-FES_{elect\_waste}
    \label{eq:FED-elect-NRE}
\end{equation}


\paragraph{RES electric overcapacity due to RES variability}

The most abundant RES for the generation of electricity, solar and wind, are subject to temporal variability. Variable RES are characterized by short-term (e.g., cloudiness. day-night) and seasonal variability. A renewable mix scenario allows to partially mitigate the variablity of different RES. However, this complementarity is far from the compensation of different sources. In any region there is a certain probability of extreme combinations in the availability of natural resources. Moreover, there can be large annual variations in the availability of natural resources; for instance, the output of wind turbines in any given area can vary by up to 30\% from one year to the next.  It has been estimated that current electricity systems and grids can usually accommodate up to only 20\% electricity from renewable sources without a need for dedicated storage facilities. Thus, a certain level of: (1) storage, (2) grid development (3) overcapacity and/or (4) flexible demand management should then be considered if a high penetration RES electricity system is designed.


Overcapacities are limited by the economic profitability of the power plant (i.e. large overcapacities imply low Cp). From a net energy perspective, overcapacities tend to lower the EROI, which could  similarly affect the net energy profitability of the plant.

The RES for electricity generation can be classified as “baseload”, i.e. those sources that are able to supply a manageable (“dispachtable”) load such as hydro, 13 biomass and geothermal, and “variable” generation. The latter are characterized by differing levels of variability and limited predictability over various time scales, and include wind and solar technologies.

Overcapacity of RES energy generation is obtained from:

\begin{equation}
    Oc= PS_{RES}/FES_{RES}
    \label{eq:Overcapacity}
\end{equation}

Where $PS_{RES}$ is the Potential supply of RES sources and $FES_{RES}$ is the real final energy supply obtained.

\paragraph{RES electric total monetary investment}

The monetary investment for building new plants up to 2050 is computed following (Teske et al.,
2011). We assume the same price to new plants and repowering plants, in order to not understimate the cost. 

The costs related to the variability of RES and the need of grid development are modeled taking into account studies for wind. Grid reinforcement costs are obtained from the median value calculated in (Mills et al., 2012)  for 40 transmission studies for wind energy
in the USA.

Other monetary costs, as balancing costs, are also introduced into the model. We assume here similar costs for the combined variable
renewable producers -solar and wind-, extrapolating the cost until it reaches a maximum of 8
euros/MWh (7.6 US 1995\$/MWh) at 50\% of total electricity share.

\enric{Detail monetary investments}

\paragraph{Electricity generation from CHP plants}

The development of these plants is estimated as a function of the remaining commercial heat demand that is not covered by renewables sources. Tendencies are maintained. Once commercial heat produced in CHP plants is estimated to cover the demand, CHP plants efficiencies are used to obtain the electricity produced
in each of these plants.
